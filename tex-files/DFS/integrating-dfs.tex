Every node running on the Catalyst network must also be running the DFS module. This allows them on the most basic level to access the current state of the ledger and the validate previous updates to the ledger thereby allowing them to sync with the current ledger state. This is required as it is necessary for the all nodes on the network at the very least to know the current state of the ledger in order to send transactions. \\

The primary difference between the native IPFS protocol and the implementation utilised on Catalyst is how the peer IDs are created.  While on the IPFS protocol the identifier for nodes is selected randomly, on Catalyst the nodes on the network will each have their own individual peer identifier\cite{BytesExtentions} made up of:

\begin{itemize}
\item IP address
\item Port number
\item Public key
\end{itemize}

% Can't you create your own ID in IPFS? Why not just mandate IPFS interoperability with your own peer names? Are you forking the rest of IPFS's functionality?

This allows user to be able to identify who on the network is holding specific files as each peer identifier for a node will be unique. This means that a peer on the network will have a clear target for where they can retrieve given information, furthermore it means that nodes claiming to hold specific files and information are accountable for being able to distribute these files. Through the use of Catalyst peer identifiers within the DHTs peers can make informed decisions on who they retrieve files from, for example if a node has been marked as malicious due to actions within the network, they will also not be trusted within the DFS.
