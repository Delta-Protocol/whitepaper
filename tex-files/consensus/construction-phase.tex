The first phase of the Catalyst consensus algorithm is the Construction Phase. In this phase, the selected producer nodes $P$ calculate their proposed ledger state update. This is done by aggregating and validating all transactions that have occurred, these transactions are ordered and producers will know the target number of well formed transactions to be included in the update. These transactions, assuming their validity, are represented by the list of digital signatures extracted the transactions, from which they can create a hash of the update. This hash digest represents what they believe to be the correct update and is broadcast to the other producer nodes for that cycle. Assuming the collision free nature of hash functions, the only mechanism for multiple producer nodes to have the same local ledger state update is for both of them to use the same set of transactions. \\

Each producer $p$ in the set of producers $P$ follows the same protocol. The construction phase begins with producer $p$ selecting the set of $n$ transactions that are to be integrated into the next ledger update. This $n$ transactions is determined by the network meaning that it is uniform for all producers and can be considered the difficulty of the block generation. This difficulty will decrease at times of low transaction throughput and increase at times of higher demand from the network. This also means that all selected producers know how many valid transactions they should attempt to add to a ledger update. The set of transactions $T$ selected by $p$ to process is made up of a set of transactions $t$. These transaction are used to create $p$'s local ledger state update $u$. \\

This set of transactions will be ordered as and when they a received by a producer through the following process:

\begin{itemize} 
\item A hash of each transaction is created using the salt $\sigma$ which is created utilizing a pseudo-random number generator using the previous ledger state update.
\begin{center}
$H = \mathcal{H}[t~||~\sigma]$
\end{center}
\item This creates a pair ($t,H$).
\item The list of transactions $t$ are sorted alphanumerically according to $H$.
\end{itemize}

Once the threshold according to the required number of transactions has been met $p$ creates a hash tree $d$. This is to store the signatures that are extracted from each transaction in $T$ and is used as an excerpt of the ledger update. 

\begin{enumerate}


\item $d$ will be ordered equivalently to the list of selected transactions i.e. each $d$ will map in order to a corresponding transaction $t$ from $T$. 

\item The producer $p$ then extracts the transaction fee value from each transaction in $T$ to create $v$ which is the total sum of all transaction fees.

\item The local ledger state update $u$ for producer $p$ can then be calculated. This is done by concatenating (denoted by $||$) the hash tree $d$ with the producer $p$'s peer ID denoted by PId and a hash digest is created as:

\begin{center}
$u = \mathcal{H}(d~||~PId)$
\end{center}

This summary of transactions $u$ is then concatenated with $p$'s $PId$ to create:

\begin{center}
$h = u ~||~PId$
\end{center}

\item $p$ is then broadcasts a special signed transaction to the RANDAO with $h$ as the message.
\end{enumerate}


Producer $p$ also collects other producers' partial ledger update values. At most they will collect $P-1$ values. Optimally every producer in $P$ will receive the same set of transactions, so for every $p$ in $P$ they will have the same partial ledger update $u$. However this is unlikely due to all transactions not being received by a small group of nodes, however because each producer node knows how many . Equally they may not hold $G$ where $G = h_1,...,h_P$, meaning they may not receive a proposed update from all candidates.
