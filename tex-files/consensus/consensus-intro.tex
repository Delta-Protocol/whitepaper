Distributed networks have no single point of trust to determine the validity of transactions, so consistency must be ensured by other methods. Typically this requires a majority of the network's participants to agree on a particular update of the ledger and the changes to account balances held on the ledger. Blockchain technologies generally employ Proof-of-Work (PoW) and occasionally Proof-of-Stake (PoS) mechanisms in order to gain consensus across a network. However, these methods are prone to increasing centralisation at scale, as well as high energy consumption (in the case of PoW). Other networks employ a small amount of trusted nodes that ensure the validity of transactions, but this is also highly centralised and almost as fallible as the single point of failure systems that DLT endeavors to avoid. \\

Catalyst integrates a newly designed consensus mechanism, based on Probabilistic Byzantine Fault Tolerance (PBFT). However, Catalyst gains consensus through a collaborative and fair mechanism by voting rather than through a competitive process, meaning that all honest work performed by nodes on the network benefits the security of the network and that all successful participating nodes are rewarded equally. The consensus for Catalyst can therefore be more accurately described as a PBFT voting mechanism. For each ledger cycle a random selection of worker nodes are selected, the nodes become the producers for a cycle or number of cycles. The producer nodes perform work in the form of compiling and validating transaction thereby extracting a ledger state change for that cycle. \\

The protocol is split into four distinct phases:

\begin{itemize}

\item Construction Phase - Producer nodes that have been selected create what they believe to be the correct update of the ledger. They then distribute this proposed ledger update in the form of a hash digest.
\item Campaigning Phase - Producer nodes designate and declare what they propose to be the ledger state update as determined by the values collected from other producer nodes in the producer pool.
\item Voting Phase - Producer nodes vote for what they believe to be the most popular ledger state update as determined by the producer pool according to the candidates for the most popular update from the other producer nodes.
\item Synchronisation Phase - The producers who have computed the correct ledger update can broadcast this update to the rest of the network.

\end{itemize}

This section is based on the work set out in \cite{catalystresearch}, where the original research into the creation of a new consensus mechanism is laid out. The various parameters and thresholds mentioned in this chapter and their impact on the levels of security and confidence in the successful production of a ledger state update are discussed in the following paper \cite{catalystresearch2}.
